\documentclass{beamer}
\usetheme{metropolis}
%\setsansfont[BoldFont={Fira Sans SemiBold}]{Fira Sans Book}
%\setsansfont{Fontin}
%\setsansfont{Gillius ADF No2}
%\setsansfont{Phetsarath OT}
\setsansfont{Source Sans Pro}
\setmonofont{Source Code Pro}

\hypersetup{colorlinks=true,
            linkcolor=mRustLightOrange,
            menucolor=mRustLightOrange,
            pagecolor=mRustLightOrange,
            urlcolor=mRustLightOrange}
\usepackage{csquotes}
\usepackage{comment}
\usepackage{xcolor}
\usepackage{minted}

\newfontfamily\codefont{Source Code Pro}
\newcommand\code[1]{\,{\color[HTML]{884400}#1}\,}
\newcommand\source[1]{$\rightarrow$ via #1}

\title{Rust's advanced type system}
\date{\today}
\author{Lukas Prokop}
\institute{RustGraz community\vfill\hfill\includegraphics[height=2cm]{images/rustacean-orig-noshadow.png}}
\begin{document}
\maketitle

\section{Prologue}

\begin{frame}[standout]
  How is \mintinline{rust}{assert_eq!} implemented? Does it typecheck at compile-time?
\end{frame}

\begin{frame}[fragile]{macros}
  \begin{minted}[fontsize=\scriptsize]{rust}
macro_rules! assert_eq {
  ($left:expr, $right:expr) => ({
    match (&$left, &$right) {
      (left_val, right_val) => {
        if !(*left_val == *right_val) {
          // The reborrows below are intentional.
          // Without them, the stack slot for the
          // borrow is initialized even before the
          // values are compared, leading to a
          // noticeable slow down.
          panic!(r#"assertion failed: `(left == right)`
left: `{:?}`,
right: `{:?}`"#, &*left_val, &*right_val)
        }
      }
    }
  });
  \end{minted}
\end{frame}


\begin{frame}[fragile]{macros}
  \begin{minted}[fontsize=\scriptsize]{rust}
  ($left:expr, $right:expr,) => ({
    $crate::assert_eq!($left, $right)
  });
  ($left:expr, $right:expr, $($arg:tt)+) => ({
    match (&($left), &($right)) {
      (left_val, right_val) => {
        if !(*left_val == *right_val) {
          // The reborrows below are intentional.
          // Without them, the stack slot for the
          // borrow is initialized even before the
          // values are compared, leading to a
          // noticeable slow down.
          panic!(r#"assertion failed: `(left == right)`
left: `{:?}`,
right: `{:?}`: {}"#, &*left_val, &*right_val,
                 $crate::format_args!($($arg)+))
        }
      }
    }
  });
}
  \end{minted}
\end{frame}

\section{Dialogue}

\begin{frame}[fragile]{Recap: traits}
  \begin{itemize}
    \item Semantically like a contract. Methods and constants.
    \item No subtyping, no inheritance, inspired by Haskell typeclasses.
    \item Trait can be implemented iff trait or type is local (trait coherence)
    \item implementation in \mintinline{rust}{impl Trait for Type} block
  \end{itemize}
  \begin{minted}{rust}
use std::os::unix::net;
use std::fs;

trait Sendable {
    fn to_socket(&self, s: net::UnixStream);
    fn to_file(&self, l: fs::File);
}
\end{minted}
\end{frame}

\begin{frame}[fragile]{Recap: generics}
  \begin{itemize}
    \item Replace a type by a \emph{type argument}
    \item Actual type will be inserted upon instantiation
    \item Implemented with monomorphisation
    \item Trait bounds to require implementation of a trait
  \end{itemize}
  \begin{minted}{rust}
fn send<T: Sendable>(sender: T, req: Request) {
    if req.header.dest() == Resources::FILE {
      sender.to_file(req.log_file);
    } else {
      sender.to_socket(req.socket);
    }
}
\end{minted}
\end{frame}

\begin{frame}[standout]
  \mintinline{rust}{Trait1} requires implementation of \mintinline{rust}{Trait2}
\end{frame}

\begin{frame}[fragile]{Require implementation of another trait}
  \begin{itemize}
    \item If \mintinline{rust}{Trait1} is implemented, then \mintinline{rust}{Trait2} must be implemented
    \item \mintinline{rust}{trait Trait1: Trait2}
    \item Concept that comes closest to OOP subtyping, but no inheritance
  \end{itemize}
  \begin{minted}{rust}
trait JSONToFile: JSONSerializable {
  fn json_to_file(&self, s: fs::File)
    -> io::Result<()>;
}
\end{minted}
\end{frame}

\begin{frame}[fragile]{Require trait implementation (example)}
  \begin{minted}{rust}
use std::{io,fs};
use std::io::{BufWriter,Write};
use std::fs::File;

trait JSONSerializable {
  fn to_json(&self) -> Vec<u8>;
}

trait JSONToFile: JSONSerializable {
  fn json_to_file(&self, s: fs::File)
    -> io::Result<()>;
}    
\end{minted}
\end{frame}

\begin{frame}[fragile]{Require trait implementation (example)}
  \begin{minted}[fontsize=\small]{rust}
struct MyType { val: u32 }

impl JSONSerializable for MyType {
  fn to_json(&self) -> Vec<u8> {
    let mut bytes = Vec::new();
    for v in br#"{"type":"int", "value":"#.iter() {
      bytes.push(*v);
    }
    for v in format!("{}", self.val).bytes() {
      bytes.push(v);
    }
    bytes.push(b'}');
    bytes
  }
}

\end{minted}
\end{frame}

\begin{frame}[fragile]{Require trait implementation (example)}
  \begin{minted}{rust}
impl JSONToFile for MyType {
  fn json_to_file(&self, fd: fs::File)
    -> io::Result<()>
  {
    let mut writer = BufWriter::new(fd);
    writer.write(self.to_json().as_slice())?;
    Ok(())
  }
}
\end{minted}
\end{frame}

\begin{frame}[fragile]{Require trait implementation (example)}
  \begin{minted}{rust}
fn write_json<T: JSONToFile>
  (w: T, filepath: &str) -> io::Result<()>
{
  let fd = File::create(filepath)?;
  w.json_to_file(fd)
}

fn main() -> io::Result<()> {
  write_json(
    MyType { val: 42u32 },
    "example.json"
  )
}
\end{minted}
\end{frame}

\begin{frame}[standout]
  Which operators are overwritable?
\end{frame}

\begin{frame}[fragile]{Non-overloadable operators}
  There is a defined set of operator traits.
  The following operators cannot be overwritten:
  \begin{itemize}
    \item \mintinline{rust}{?} for error handling
    \item \mintinline{rust}{||} as \emph{lazy boolean or}
    \item \mintinline{rust}{&&} as \emph{lazy boolean and}
    \item \mintinline{rust}{=} as \emph{assignment operator}
    \item \mintinline{rust}{&v &&v &&&v} \dots\ to get a reference
    \item \mintinline{rust}{&mut v  &&mut v} \dots\ to get a mutable reference
  \end{itemize}
\end{frame}

\begin{frame}[fragile]{n-ary references}
  \begin{minted}{rust}
fn print_number(s: &&u32) {
  println!("{}", **s);  // prints “42”
}

fn main() {
  print_number(&&42);
}
\end{minted}
  \phantom{Does it compile? Yes, \href{https://stackoverflow.com/q/28519997}{auto-dereferencing}.}
\end{frame}

\begin{frame}[fragile]{n-ary references}
  \begin{minted}{rust}
fn print_number(s: &&u32) {
  println!("{}", s);  // w/o “**”
}

fn main() {
  print_number(&&42);
}
\end{minted}
  Does it compile? \pause{Yes, \href{https://stackoverflow.com/q/28519997}{auto-dereferencing}.}
\end{frame}

\begin{frame}[fragile]{n-ary references in C}
  \begin{minted}{C}
#include <stdio.h>
#include <stdint.h>

void print_number(uint32_t **s) {
  printf("%u", **s);
}
int main() {
  print_number(&&42);
  return 0;
}
\end{minted}
  Does it compile? \pause{No.}
\end{frame}

\begin{frame}[fragile]{n-ary references in C}
  \begin{minted}{text}
main.c:8:18: error: expected identifier
  print_number(&&42);
                 ^
1 error generated.
\end{minted}
  Let's consider one level less. The error message becomes more explicit.
\end{frame}

\begin{frame}[fragile]{n-ary references in C}
  \begin{minted}{C}
#include <stdio.h>
#include <stdint.h>

void print_number(uint32_t *s) {
  printf("%u", *s);
}
int main() {
  print_number(&42);
  return 0;
}
\end{minted}
Does it compile? No.
\end{frame}

\begin{frame}[fragile]{n-ary references}
  \begin{minted}{text}
main.c:8:16: error: cannot take the address of
                    an rvalue of type 'int'
  print_number(&42);
                ^~~
1 error generated.
\end{minted}
\end{frame}

\begin{frame}[fragile]{n-ary references}
  \begin{minted}{C}
#include <stdio.h>
#include <stdint.h>

void print_number(uint32_t *s) {
  printf("%u", *s);
}
int main() {
  uint32_t val = 42;
  print_number(&val);
  return 0;
}
\end{minted}
  Does it compile? \pause{Yes.}
\end{frame}

\begin{frame}[fragile]{n-ary references}
  \textbf{Summary:}
  \begin{itemize}
    \item In rust, you can take references to constant values.
    \item In C, you cannot (unless you assign them).
  \end{itemize}
\end{frame}

\begin{frame}[fragile]{Overloadable operators}
  This the defined (and exhaustive) set of operator traits.
  \begin{itemize}
    \item implement \mintinline{rust}{std::ops::Neg} for \mintinline{rust}{-}
    \item implement \mintinline{rust}{std::ops::Not} for \mintinline{rust}{!}
    \item implement \mintinline{rust}{std::ops::Add} for \mintinline{rust}{+}
    \item implement \mintinline{rust}{std::ops::Sub} for \mintinline{rust}{-}
    \item implement \mintinline{rust}{std::ops::Mul} for \mintinline{rust}{*}
    \item implement \mintinline{rust}{std::ops::Div} for \mintinline{rust}{/}
    \item implement \mintinline{rust}{std::ops::Rem} for \mintinline{rust}{%}
  \end{itemize}
\end{frame}

\begin{frame}[fragile]{Overloadable operators}
  \begin{itemize}
    \item implement \mintinline{rust}{std::ops::BitAnd} for \mintinline{rust}{&}
    \item implement \mintinline{rust}{std::ops::BitOr} for \mintinline{rust}{|}
    \item implement \mintinline{rust}{std::ops::BitXor} for \mintinline{rust}{^}
    \item implement \mintinline{rust}{std::ops::Shl} for \mintinline{rust}{<<}
    \item implement \mintinline{rust}{std::ops::Shr} for \mintinline{rust}{>>}
    \item implement \mintinline{rust}{std::cmp::PartialEq::eq} for \mintinline{rust}{==}
    \item implement \mintinline{rust}{std::cmp::PartialEq::ne} for \mintinline{rust}{!=}
    \item implement \mintinline{rust}{std::cmp::PartialOrd::gt} for \mintinline{rust}{>}
    \item implement \mintinline{rust}{std::cmp::PartialOrd::lt} for \mintinline{rust}{<}
    \item implement \mintinline{rust}{std::cmp::PartialOrd::ge} for \mintinline{rust}{>=}
    \item implement \mintinline{rust}{std::cmp::PartialOrd::le} for \mintinline{rust}{<=}
  \end{itemize}
\end{frame}

\begin{frame}[fragile]{Overloadable operators}
  \begin{itemize}
    \item implement \mintinline{rust}{std::ops::AddAssign} for \mintinline{rust}{+=}
    \item implement \mintinline{rust}{std::ops::SubAssign} for \mintinline{rust}{-=}
    \item implement \mintinline{rust}{std::ops::MulAssign} for \mintinline{rust}{*=}
    \item implement \mintinline{rust}{std::ops::DivAssign} for \mintinline{rust}{/=}
    \item implement \mintinline{rust}{std::ops::RemAssign} for \mintinline{rust}{%=}
    \item implement \mintinline{rust}{std::ops::BitAndAssign} for \mintinline{rust}{&=}
    \item implement \mintinline{rust}{std::ops::BitOrAssign} for \mintinline{rust}{|=}
    \item implement \mintinline{rust}{std::ops::BitXorAssign} for \mintinline{rust}{^=}
    \item implement \mintinline{rust}{std::ops::ShlAssign} for \mintinline{rust}{<<=}
    \item implement \mintinline{rust}{std::ops::ShrAssign} for \mintinline{rust}{>>=}
  \end{itemize}
\end{frame}

\begin{frame}[fragile]{Overloadable operators}
  \begin{itemize}
    \item implement \mintinline{rust}{std::ops::Index} for indexing \mintinline{rust}{v[i]}
    \item implement \mintinline{rust}{std::ops::IndexMut} for mutable indexing
    \item implement \mintinline{rust}{std::ops::Deref} for \mintinline{rust}{* ** ***} \dots
    \item implement \mintinline{rust}{std::ops::DerefMut} for \mintinline{rust}{*mut **mut ***mut} \dots
  \end{itemize}
\end{frame}

\begin{frame}[fragile]{Deref example}
  \textbf{Goal:}
  \begin{itemize}
    \item Let \texttt{v} be value 42 (wrapped by custom type).
    \item \mintinline{rust}{v} is represented as \mintinline{text}{42}
    \item \mintinline{rust}{&v} is represented as \mintinline{text}{(42)}
    \item \mintinline{rust}{&&v} is represented as \mintinline{text}{((42))}
    \item \mintinline{rust}{&&&v} is represented as \mintinline{text}{(((42)))} \dots
  \end{itemize}
  \pause
  \textbf{Problem:}
  \begin{itemize}
    \item We cannot overload \mintinline{rust}{&v}, but \mintinline{rust}{*v}
    \item Let \mintinline{rust}{v} be \mintinline{text}{((((((((42))))))))}
    \item Let \mintinline{rust}{*v} be \mintinline{text}{((((((42))))))}
    \item Let \mintinline{rust}{**v} be \mintinline{text}{(((((42)))))} \dots
  \end{itemize}
\end{frame}

\begin{frame}[fragile]{Deref example}
  \begin{minted}[fontsize=\scriptsize]{rust}
use std::fmt;
use std::ops::Deref;

struct Wrapped<T: fmt::Display> {
  value: T,
  depth: usize,
}

impl<T: fmt::Display> fmt::Display for Wrapped<T> {
  fn fmt(&self, f: &mut fmt::Formatter) -> fmt::Result {
    write!(f, "{}{}{}", "(".repeat(self.depth),
           self.value, ")".repeat(self.depth))
  }
}
\end{minted}
\end{frame}

\begin{frame}[fragile]{Deref example}
  \begin{minted}[fontsize=\scriptsize]{rust}
impl<T: fmt::Display> Deref for Wrapped<T> {
  type Target = T;

  fn deref(&self) -> &Self::Target {
    &Wrapped { value: self.value, depth: self.depth - 1 }
  }
}

fn main() {
    let v = Wrapped{ value: 1, depth: 8 };
    println!("{}", v);
    println!("{}", ****v);
    println!("{}", ********v);
}
\end{minted}
  Does it compile? \pause No.
\end{frame}

\begin{frame}[fragile]{Deref example}
  \begin{minted}[fontsize=\scriptsize]{text}
error[E0614]: type `{integer}` cannot be dereferenced
  --> src/main.rs:27:22
   |
27 |     println!("{}", ****v);
   |                      ^^^

error[E0614]: type `{integer}` cannot be dereferenced
  --> src/main.rs:28:26
   |
28 |     println!("{}", ********v);
   |                          ^^^
\end{minted}
  Why?
\end{frame}

\begin{frame}[fragile]{Deref example (hint for error)}
  \begin{minted}{rust}
struct Wrapped<T: fmt::Display> {
  value: T,
  depth: usize,
}

impl<T: fmt::Display> Deref for Wrapped<T> {
  type Target = T;

  fn deref(&self) -> &Self::Target {
    &Wrapped { value: self.value,
               depth: self.depth - 1 }
  }
}
\end{minted}
  \pause \texttt{T} is \mintinline{rust}{u32}, thus \mintinline{rust}{Self::Target} as well.
\end{frame}


\begin{frame}[fragile]{Deref example}
  \textbf{My learning process:}
  \begin{itemize}
  	\item Where can we store the depth information?
    \item \mintinline{rust}{fn deref(&self) -> &Self::Target} uses \mintinline{rust}{&self} (c.f. \mintinline{rust}{&mut self}). Does not permit mutation.
    \item We also cannot create new object, because where do we store it? (switch to heap objects like Rc would be possible)
  \end{itemize}
\end{frame}

\begin{frame}[fragile]{Deref example (fixed)}
  \begin{minted}{rust}
fn main() {
    let v1 = Wrapped{ value: 42, depth: 0 };
    let v2 = Wrapped{ value: &v1, depth: 1 };
    let v3 = Wrapped{ value: &v2, depth: 2 };
    let v4 = Wrapped{ value: &v3, depth: 3 };
    println!("{}", v1);  // "42"
    println!("{}", v2);  // "(42)"
    println!("{}", *v2); // "42"
}
\end{minted}
\end{frame}

\begin{frame}[fragile]{Deref example (fixed)}
  \begin{minted}[fontsize=\scriptsize]{rust}
use std::fmt;
use std::ops::Deref;

struct Wrapped<T: fmt::Display> {
  value: T,
  depth: usize,
}

impl<T: fmt::Display> Deref for Wrapped<T> {
  type Target = T;

  fn deref(&self) -> &Self::Target {
    &Wrapped { value: self.value,
               depth: self.depth - 1 }
  }
}
\end{minted}
\end{frame}

\begin{frame}[fragile]{Deref example (fixed)}
  \begin{minted}[fontsize=\scriptsize]{rust}
impl<T: fmt::Display> fmt::Display for Wrapped<T> {
    fn fmt(&self, f: &mut fmt::Formatter)
       -> fmt::Result {
        write!(f,
               "{}{}{}",
               "(".repeat(self.depth),
               self.value,
               ")".repeat(self.depth))
    }
}

impl<T: fmt::Display> Deref for Wrapped<T> {
    type Target = T;

    fn deref(&self) -> &Self::Target {
        &self.value
    }
}
\end{minted}
\end{frame}

\begin{frame}[standout]
  Trait objects
\end{frame}

\begin{frame}[fragile]{Method dispatch}
  \textbf{Cliff hanger, last time}: Can we take references to traits?

  Let T be a trait. We call \mintinline{rust}{T.method()}. \\
  Where do we find the implementation of \texttt{method}?
  \begin{itemize}
    \item \textbf{static dispatch:} monomorphization, like C++ templates, preferred dispatch
    \item \textbf{dynamic dispatch:} trait objects
  \end{itemize}

  One application example for dynamic dispatch: What about a vector of objects implementing a trait; \mintinline{rust}{Vec<Trait>}?
  \begin{itemize}
    \item \textbf{static dispatch:} wrap each possible type with \mintinline{rust}{enum}, unextensible to external types
    \item \textbf{dynamic dispatch:} trait object
  \end{itemize}
\end{frame}

\begin{frame}[fragile]{Monomorphisation example}
  \begin{minted}{rust}
struct Wrapped { val: u32 }

trait Numeric {
  fn as_u32(&self) -> u32;
}

impl Numeric for Wrapped {
  fn as_u32(&self) -> u32 {
    self.val
  }
}
\end{minted}
\end{frame}

\begin{frame}[fragile]{Monomorphisation example}
  \begin{minted}{rust}
fn print_int<T: Numeric>(obj: T) {
    println!("{:x}", obj.as_u32());
}

fn main() {
    let v = Wrapped { val: 42 };
    print_int(v);  // "2a"
}
\end{minted}
\end{frame}

\begin{frame}[fragile]{Trial and error}
  Can we provide \mintinline{rust}{&Wrapped} for \mintinline{rust}{Numeric}?
  \begin{minted}[linenos]{rust}
fn print_int<T: Numeric>(obj: T) {
    println!("{:x}", obj.as_u32());
}

fn main() {
    let v = Wrapped { val: 42 };
    print_int(&v);
}
\end{minted}
\end{frame}

\begin{frame}[fragile]{Trial and error}
  \begin{minted}[fontsize=\scriptsize]{text}
error[E0277]: the trait bound `&Wrapped: Numeric` is not satisfied
  --> src/main.rs:20:15
   |
14 | fn print_int<T: Numeric>(obj: T) {
   |    ---------    ------- required by this bound in `print_int`
...
20 |     print_int(&v);
   |               -^
   |               |
   | the trait `Numeric` is not implemented for `&Wrapped`
   | help: consider removing the leading `&`-reference
   |
   = help: the following implementations were found:
             <Wrapped as Numeric>
\end{minted}
  \pause Apparently, we can implement \mintinline{rust}{Numeric} for \mintinline{rust}{&Wrapped} as well.
\end{frame}

\begin{frame}[fragile]{Trial and error}
  \begin{minted}[fontsize=\scriptsize]{rust}
struct Wrapped { val: u32 }
trait Numeric {
  fn as_u32(&self) -> u32;
}

impl Numeric for Wrapped {
  fn as_u32(&self) -> u32 { self.val }
}
impl Numeric for &Wrapped {
  fn as_u32(&self) -> u32 { (**self).val }
}

fn print_int<T: Numeric>(obj: T) {
  println!("{:x}", obj.as_u32());
}
fn main() {
  let v = Wrapped { val: 42 };
  print_int(&v); // "2a"
}
\end{minted}
\end{frame}

\begin{frame}[fragile]{Trial and error}
  What about \mintinline{rust}{&Wrapped} \emph{and} \mintinline{rust}{&Numeric}?
  \begin{minted}[linenos]{rust}
fn print_int<T: &Numeric>(obj: T) {
    println!("{:x}", obj.as_u32());
}

fn main() {
    let v = Wrapped { val: 42 };
    print_int(&v);
}
\end{minted}
\end{frame}

\begin{frame}[fragile]{Trial and error}
  \begin{minted}[fontsize=\scriptsize]{text}
error: expected one of `!`, `(`, `,`, `=`,
       `>`, `?`, `for`, lifetime, or path, found `&`
  --> src/main.rs:14:17
   |
14 | fn print_int<T: &Numeric>(obj: T) {
   |                 ^ expected one of 9 possible tokens

error: aborting due to previous error
\end{minted}
\end{frame}

\begin{frame}[fragile]{Trial and error}
  \textbf{Recap:} Let's switch syntax. Can we use \mintinline{rust}{Numeric} as type?
  \begin{minted}[linenos]{rust}
fn print_int(obj: Numeric) {
    println!("{:x}", obj.as_u32());
}

fn main() {
    let v = Wrapped { val: 42 };
    print_int(v);
}
\end{minted}
  \pause
  No, we need to use the \mintinline{rust}{impl} keyword!
\end{frame}

\begin{frame}[fragile]{Trial and error}
  \textbf{Recap:} use \mintinline{rust}{impl}! This is the equivalent syntax to \mintinline{rust}{fn print_int<T: Numeric>(obj: T)}.
  \begin{minted}[linenos]{rust}
fn print_int(obj: impl Numeric) {
    println!("{:x}", obj.as_u32());
}

fn main() {
    let v = Wrapped { val: 42 };
    print_int(v);
}
\end{minted}
\end{frame}

\begin{frame}[fragile]{Trial and error}
  But it \textbf{does work} if we use \mintinline{rust}{&Numeric} as type!
  \begin{minted}[linenos]{rust}
fn print_int(obj: &Numeric) {
    println!("{:x}", obj.as_u32());
}

fn main() {
    let v = Wrapped { val: 42 };
    print_int(&v);   // "2a"
}
\end{minted}
\end{frame}

\begin{frame}[standout]
  \dots\ so, what is a trait object?
\end{frame}

\begin{frame}[fragile]{Trait objects}
  \textbf{Idea:}
  \begin{itemize}
    \item We generate an separate object from an object maintaining pointers to the actual implementation of the methods specified in the trait (and \emph{only those})!
    \item Exactly like Golang's function call with argument of interface type
    \item Similar to C++'s vtable
    \item Runtime overhead, no inlining of function calls
  \end{itemize}
\end{frame}

\begin{frame}[fragile]{Trait objects}
  \textbf{Syntax and history:}
  \begin{itemize}
    \item \mintinline{rust}{dyn} keyword: \mintinline{rust}{dyn Trait} is a type referring to any trait object implementing \texttt{Trait}
    \item Since \href{https://blog.rust-lang.org/2018/06/21/Rust-1.27.html}{Rust 1.27}. Is \texttt{Foo} a struct or a trait?
  \end{itemize}
  \begin{tabular}{lll}
\mintinline{rust}{Box<Foo>} & became & \mintinline{rust}{Box<dyn Foo>} \\
\mintinline{rust}{&Foo    } & became & \mintinline{rust}{&dyn Foo} \\
\mintinline{rust}{&mut Foo} & became & \mintinline{rust}{&mut dyn Foo}
\end{tabular}
\end{frame}

\begin{frame}[fragile]{Trait objects}
  \textbf{Requirements} for traits to generate trait objects (\enquote{object safe}):
  \begin{itemize}
    \item All return types must not be Self.
    \item No generic type parameters.
  \end{itemize}
\end{frame}

\begin{frame}[fragile]{Trial and error}
  Standard library's \mintinline{rust}{Clone} trait is \textbf{not} object-safe:
  \begin{minted}{rust}
pub trait Clone {
    fn clone(&self) -> Self;
}
\end{minted}
  \pause Thus, \mintinline{rust}{dyn Clone} is not permitted.
\end{frame}

\begin{frame}[fragile]{Trait object example}
  \begin{minted}{rust}
trait Named {
    fn name(&self) -> String;
}

struct Student { name: String }
struct Teacher { name: String }
  \end{minted}
\end{frame}

\begin{frame}[fragile]{Trait object example}
  \begin{minted}{rust}
impl Named for Student {
    fn name(&self) -> String {
        let mut s = String::from("student ");
        s.push_str(&self.name);
        s
    }
}

impl Named for Teacher {
    fn name(&self) -> String {
        let mut s = String::from("teacher ");
        s.push_str(&self.name);
        s
    }
}
  \end{minted}
\end{frame}

\begin{frame}[fragile]{Trait object example}
  \begin{minted}[fontsize=\scriptsize]{rust}
fn main() {
  let s1 = Student { name: String::from("Lukas") };
  let s2 = Student { name: String::from("Anita") };
  let t1 = Teacher { name: String::from("Sensei") };
  println!("{}\n{}\n{}", s1.name(), s2.name(), t1.name());
}
  \end{minted}
  \begin{minted}{text}
student Lukas
student Anita
teacher Sensei
  \end{minted}
\end{frame}

\begin{frame}[fragile]{Trait object example}
  \begin{minted}{rust}
struct Container {
    elements: Vec<dyn Named>,
}
  \end{minted}
  \begin{minted}[fontsize=\scriptsize]{text}
error[E0277]: the size for values of type
  `(dyn Named + 'static)` cannot be known
  at compilation time
  --> src/main.rs:25:5
   |
25 |     elements: Vec<dyn Named>,
   |     ^^^^^^^^^^^^^^^^^^^^^^^^
   |     doesn't have a size known at compile-time
   |
   = help: the trait `std::marker::Sized` is not implemented
          for `(dyn Named + 'static)`
   = note: required by `std::vec::Vec`
  \end{minted}
\end{frame}

\begin{frame}[fragile]{Trait object example}
  \begin{minted}{rust}
struct Container {
  elements: Vec<Box<dyn Named>>,
}

fn main() {
  let c = Container {
    elements: vec![
      Box::new(s1), Box::new(s2),
       Box::new(t1)
  ]};
  
  println!("{}", c.elements[0].name());
  // "student Lukas"
}
\end{minted}
\end{frame}

\begin{frame}[fragile]{Trait object example}
  \begin{minted}{rust}
struct Container<T: Named> {
    elements: Vec<Box<T>>,
}

fn main() {
  let c = Container {
    elements: vec![
      Box::new(s1), Box::new(s2),
       Box::new(t1)
  ]};
  println!("{}", c.elements[0].name());
}
\end{minted}
  What's the difference?
\end{frame}

\begin{frame}[fragile]{Trait object example}
  \begin{minted}[fontsize=\scriptsize]{text}
error[E0308]: mismatched types
  --> src/main.rs:35:61
   |
35 | elements: vec![Box::new(s1), Box::new(s2), Box::new(t1)]
   |   expected struct `Student`, found struct `Teacher` ^^

error: aborting due to previous error
  \end{minted}
\end{frame}

\begin{frame}[fragile]{Trait object example}
  \begin{minted}[fontsize=\scriptsize]{text}
error[E0308]: mismatched types
  --> src/main.rs:35:61
   |
35 | elements: vec![Box::new(s1), Box::new(s2), Box::new(t1)]
   |   expected struct `Student`, found struct `Teacher` ^^

error: aborting due to previous error
  \end{minted}
\end{frame}

\begin{frame}[fragile]{Trait object coercion}
  \begin{itemize}
    \item Casts for trait objects are possible, but rarely necessary
    \item Use \mintinline{rust}{as} keyword for coercion
    \item \mintinline{rust}{&obj as &Trait}
    \item Allows to tests for object safety
  \end{itemize}
  \begin{minted}[fontsize=\scriptsize]{rust}
let v = vec![1, 2, 3];
let o = &v as &Clone;
  \end{minted}
\end{frame}

\begin{frame}[standout]
  The \emph{type keyword} or \emph{associated types}
\end{frame}

\begin{frame}[fragile]{Associated types}
  \begin{minted}{rust}
trait Graph<N, E> {
    fn has_edge(&self, &N, &N) -> bool;
    fn edges(&self, &N) -> Vec<E>;
    // Etc.
}
  \end{minted}
  A Graph generic over any node type and edge type.
\end{frame}

\begin{frame}[fragile]{Associated types}
  \begin{minted}{rust}
fn distance<N, E, G: Graph<N, E>>(
  graph: &G, start: &N, end: &N
) -> u32 {
  // implementation of distance function
}
  \end{minted}
\end{frame}

\begin{frame}[fragile]{Associated types}
  \begin{minted}{rust}
trait Graph {
  type N;
  type E;

  fn has_edge(&self, &Self::N, &Self::N) -> bool;
  fn edges(&self, &Self::N) -> Vec<Self::E>;
}
  \end{minted}
\end{frame}

\begin{frame}[fragile]{Associated types}
  \begin{itemize}
    \item Associated types are declared with the \mintinline{rust}{type} keyword within a trait
    \item Binds a type to some instance of a graph
    \item In our example: Graph is a trait with two associated types \emph{N} and \emph{E}
  \end{itemize}
\end{frame}


\begin{frame}[fragile]{Associated types}
  \begin{itemize}
    \item Associated types are declared with the \mintinline{rust}{type} keyword within a trait
    \item Binds a type to some instance of a graph
    \item In our example: Graph is a trait with two associated types \emph{N} and \emph{E}
  \end{itemize}
\end{frame}

\begin{frame}[fragile]{Associated types}
  \begin{minted}[fontsize=\scriptsize]{rust}
struct Node;
struct Edge;
struct MyGraph;

impl Graph for MyGraph {
  type N = Node;
  type E = Edge;

  fn has_edge(&self, n1: &Node, n2: &Node)
     -> bool
  {
    true
  }

  fn edges(&self, n: &Node)
    -> Vec<Edge>
  {
    Vec::new()
  }
}
  \end{minted}
\end{frame}

\begin{frame}[fragile]{Associated types}
  \textbf{Associated types and coercion into a trait object:}

  \begin{minted}{rust}
let graph = MyGraph;
let obj = Box::new(graph) as Box<Graph>;
  \end{minted}

  \begin{minted}[fontsize=\scriptsize]{text}
error: the value of the associated type `E`
       (from the trait `main::Graph`) must
       be specified [E0191]
let obj = Box::new(graph) as Box<Graph>;
          ^~~~~~~~~~~~~~~~~~~~~~~~~~~~~
24:44 error: the value of the associated type `N`
             (from the trait `main::Graph`) must
             be specified [E0191]
let obj = Box::new(graph) as Box<Graph>;
          ^~~~~~~~~~~~~~~~~~~~~~~~~~~~~
\end{minted}
\end{frame}

\begin{frame}[fragile]{Associated types}
  \textbf{Solution:} explicit assignment
  \begin{minted}{rust}
let graph = MyGraph;
let obj = Box::new(graph) as
  Box<Graph<N=Node, E=Edge>>;
\end{minted}
\end{frame}

\begin{frame}[fragile]{Add trait}
  \href{https://doc.rust-lang.org/src/core/ops/arith.rs.html#66}{Add trait} via stdlib implementation
  \begin{minted}[fontsize=\scriptsize]{rust}
#[lang = "add"]
#[stable(feature = "rust1", since = "1.0.0")]
#[rustc_on_unimplemented(
  on(all(_Self = "{integer}", Rhs = "{float}"), message = "cannot add a float to an integer",),
  on(all(_Self = "{float}", Rhs = "{integer}"), message = "cannot add an integer to a float",),
  message = "cannot add `{Rhs}` to `{Self}`",
  label = "no implementation for `{Self} + {Rhs}`"
)]
#[doc(alias = "+")]
pub trait Add<Rhs = Self> {
  /// The resulting type after applying the `+` operator.
  #[stable(feature = "rust1", since = "1.0.0")]
  type Output;

  /// Performs the `+` operation.
  #[must_use]
  #[stable(feature = "rust1", since = "1.0.0")]
  fn add(self, rhs: Rhs) -> Self::Output;
}
\end{minted}
\end{frame}

\begin{frame}[standout]
  TypeId
\end{frame}

\begin{frame}[fragile]{TypeId}
  \texttt{TypeId} is a crate in stdlib that allows you to reason about types (obviously at compile time) in a limited manner. One example:
  \begin{minted}[fontsize=\scriptsize]{rust}
use std::any::{Any, TypeId};

fn is_string<T: ?Sized + Any>(_s: &T) -> bool {
    TypeId::of::<String>() == TypeId::of::<T>()
}

assert_eq!(is_string(&0), false);
assert_eq!(is_string(&"cookie monster".to_string()), true);
\end{minted}
\end{frame}



% TODO interior mutability

%\begin{frame}[standout]
%  What are \mintinline{rust}{?Sized} traits?
%\end{frame}

%\begin{frame}[fragile]{\mintinline{rust}{?Sized}}
%  \begin{itemize}
%    \item TODO
%  \end{itemize}
%\end{frame}

\section{Epilogue}

\begin{frame}[fragile]{Quiz}
  \begin{description}
    %\item[What is interior mutability?] \hfill{} \\
    %  ~\uncover<2->{An immutable data structure maintains mutable objects}
    %\item[What does \mintinline{rust}{?Sized} stand for?] \hfill{} \\
    %  ~\uncover<3->{A trait bound matching non-sized types as well (i.e. types of unknown size)}
    \item[What is an associated type?] \hfill{} \\
      ~\uncover<2->{A type local to a trait}
    \item[What is dynamic dispatching?] \hfill{} \\
      ~\uncover<3->{An object is generated containing only pointers to the trait method implementations}
    \item[When do you use the \mintinline{rust}{dyn} keyword?] \hfill{} \\
      ~\uncover<4->{To refer to a trait object}
%    \item[What is referential transparency?] \hfill{} \\
%      ~\uncover<4->{A program maintains the same semantics if some variable is replaced by its value}
%    \item[When do you use the \mintinline{rust}{dyn} and \mintinline{rust}{ref} keywords?] \hfill{} \\
%      ~\uncover<5->{
%        \mintinline{rust}{dyn} adresses a reference to a trait implementation \\
%        \mintinline{rust}{ref} matches a reference in pattern matching context
%      }
  \end{description}
\end{frame}

\begin{frame}[fragile]{Quiz on macros (skipped last time)}
  \begin{description}
    \item[How can you syntactically recognize macros?] \hfill{} \\
      ~\uncover<2->{\mintinline{rust}{macro!()}}
    \item[Which two kinds of macros exist?] \hfill{} \\
      ~\uncover<3->{declarative \& procedural}
    \item[What kind of procedural macros exist?] \hfill{} \\
      ~\uncover<4->{derive macros, attribute-like macros, function-like macros}
    \item[Define macro hygiene] \hfill{} \\
      ~\uncover<5->{Local scope does not get polluted by variables introduced in macro}
    \item[Which repetition specifiers exist in macros?] \hfill{} \\
      \uncover<6->{0--infinity: \mintinline{rust}{*} \\ 0--1: \mintinline{rust}{?} \\ 1--infinity: \mintinline{rust}{+}}
  \end{description}
\end{frame}

\begin{frame}[fragile]{Next time}
  \textbf{Covid19 disclaimer:}
    Once, more than 15 people are allowed to meet and the majority is fine with it,
    we are going to schedule an offline meeting. In the following, we are going to
    hold a \emph{Hacker Jeopardy} (finally).

  \begin{tabular}{ll}
    Next meetup  & Wed, 2020/06/24 \\
    Topic        & Lifetimes, anonymous functions and modularization
  \end{tabular}
\end{frame}

\begin{frame}[fragile]{Next time}
  For the ambitious ones:
  \begin{itemize}
    \item Rust lifetimes are inspired by Cyclone's memory regions
    \item I will talk about Cyclone next time
    \item \href{https://www.usenix.org/legacy/event/usenix02/full_papers/jim/jim_html/}{Cyclone: A safe dialect of C} (2002)
    \item \href{http://cyclone.thelanguage.org/}{Cyclone homepage}
    \item Recommendation: read the Cyclone paper before next time
  \end{itemize}
\end{frame}

\begin{frame}[standout]
  Thank you!

  \includegraphics[width=40pt]{images/rustacean-flat-happy.png}
\end{frame}

\end{document}
